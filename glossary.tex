% Emulate acronym behaviours: the first time, the long version is used, then the
% short.
\newglossaryentry{ECMWF}
{ 
  name={ECMWF},
  description={European Centre for Medium-Range Weather Forecasts. It is an
  European centre providing medium-range weather and seasonal forecasts},
  first={{\glsentrylong{ECMWF}} (\glsentryname{ECMWF})},
  long={European Centre for Medium-Range Weather Forecasts}
  }
\newglossaryentry{NCAR}
{ 
  name={NCAR},
  description={National Center for Atmospheric Research. It is an federal
  organisation, located in the US, which focus on climate, weather, air
  chemistry and other related areas.},
  first={{\glsentrylong{NCAR}} (\glsentryname{NCAR})},
  long={National Center for Atmospheric Research}
}
\newglossaryentry{DWD}
{ 
  name={DWD},
  description={Deutscher Wetterdienst. German weather agency.},
  first={{\glsentrylong{DWD}} (\glsentryname{DWD})},
  long={Deutscher Wetterdienst}
}
% Normal acronyms here
\newacronym{NWP}{NWP}{Numerical Weather Prediction model}
\newacronym{CMC}{CMC}{Canadian Meteorological Center}
\newacronym{JMA}{JMA}{Japanese Meteorological Agency}
\newacronym{CFL}{CFL}{Courant-Friedrichs-Lewy}
\newacronym{CTM}{CTM}{Chemistry Transport Model}
\newacronym{MCT}{MCT}{Mod\`ele de Chimie-Transport}
\newacronym{GCM}{GCM}{Global Climate Model}
\newacronym{PPM}{PPM}{Piecewise Parabolic Method}
\newacronym{DCISL}{DCISL}{Departure Cell-Integrated Semi-Lagrangian}
\newacronym{SOM}{SOM}{Second-Order Moments}
\newacronym{GMD}{GMD}{Geoscientific Model Development}
\newacronym{flops}{flops}{floating-point operations per seconds}
\newacronym{HPC}{HPC}{High Performance Computing}
\newacronym{GPU}{GPU}{Graphics Processing Unit}
\newacronym{UPC}{UPC}{Unified Parallel C}
\newacronym{PGAS}{PGAS}{Partitioned Global Adress Space}
\newacronym{PVM}{PVM}{Parallel Virtual Machine}
\newacronym{API}{API}{Application Programming Interface}
\newacronym{OpenMP}{OpenMP}{Open Multi-Processing}
\newacronym{MPI}{MPI}{Message Passing Interface}
\newacronym{netCDF}{netCDF}{Network Common Data Form}
\newacronym{HDF}{HDF}{Hierarchical Data Format}
\newacronym{ODE}{ODE}{Ordinary Differential Equation}
\newacronym{FFT}{FFT}{Fast Fourier Transform}

%-------------------------------------------------------------------------------
% Convention: the word in the text is lower case but upper case in the glossary.
% So we use the key "text".
\newglossaryentry{biogenic}
{
  name={Biogenic},
  text={biogenic},
  description={Is said from substances produced by living organisms or
  biological processes. Examples: coal, oil, pearls, silk}
}
\newglossaryentry{stratosphere}
{
  name={Stratosphere},
  text={stratosphere},
  description={second atmosphere layer. It is located between the troposphere
  and the mesosphere. At mid-latitudes, it is situated between 18-13km and
  50km}
}
\newglossaryentry{troposphere}
{
  name={Troposphere},
  text={troposphere},
  description={first atmosphere layer from the surface of the Earth to the
  tropopause, which separates it from the stratosphere. It contains more than
  80\% of the total mass of the atmosphere. At mid-latitudes, its depth is
  around 17km}
}
\newglossaryentry{CPU}
{
  name=CPU,
  description={Central Processing Unit. It carries out instructions
  stored in a program. For that, it executes arithmetical, logical and I/O
  operations}
}
\newglossaryentry{processor}
{
  name=Processor,
  description={a multi-purpose, programmable device.  Modern processors are
  multi-cores, i.e they contains several CPUs}
}
\newglossaryentry{conformal}
{
  name={Conformal projection},
  text={conformal},
  description={this projection preserves the angles locally (i.e at the
  intersection of the lines. In particular, parallel cross meridians at right
  angles. In practice, dimensions are deformed when far from the center (they
  do not preserve areas) and as such, are used for global instead of regional
  maps. Most conformal maps have singularities}
}
\newglossaryentry{area-preserving projection}
{
  name={Area-preserving projection},
  text={area-preserving projection},
  description={this projection preserves the ratio of areas on the map and the
  areas of the globe. However, it does not preserve distances}
}
\newglossaryentry{gnomonic}
{
  name={Gnomonic projection},
  text={gnomonic projection},
  description={a projection which represents the shortest route between two
  points (greats circles) as straight lines, which makes it easy to find the
  shortest path between two points. However, distance and shape are distorted,
  except near the center of the projection}
}
\newglossaryentry{Robinson projection}
{
  name={Robinson projection},
  description={a projection, which is neither area-preserving nor conformal but
  aims to represents the whole world in a "good-looking" way. In particular, it
  aims to reduce the global distortion. Used by the National Geographic Society
  between 1988 and 1998}
}
\newglossaryentry{Lambert equal-area projection}
{
  name={Lambert equal-area projection},
  description={As an azimuthal projection, it maps a portion of the globe to
  a tangent plane. It also preserves the areas. When used for a global map,
  shapes are distorted near the boundaries}
}

\newglossaryentry{process}{
  name={Process}, 
  text={process},
  description={can be defined as an adress space and the current state of a
  program (program counter, call stack, register values).  Different processes
  do not share resources between them, contrary to threads},
  plural = {processes}
}
\newglossaryentry{thread}{
  name={Thread}, 
  text={thread},
  description={smallest set of instructions that can be managed independently by
  the operating system (program counter and call stack). It is a lightweight
  process. If several threads exist in the same process, they share resources
  (e.g memory)}
}
\newglossaryentry{stereographic}
{
  name={Stereographic projection},
  text={stereographic},
  description={this conformal projection maps the whole sphere onto a plane, except the
  projection point. For a projection center at the North pole, It computes the
  intersection of the line between the pole and the considered point with the
  plane $z=0$}
}
%\glossentry[numa]{NUMA}
%See
%\href{http://www.compunity.org/training/tutorials/2\%20Basic\_Concepts\_Parallelization.pdf}{this
%link} for a picture.
%
%\glossentry[semiimplicit]{Semi-implicit scheme}
%For a pair of equation between $a$ and $b$, use $a^n$ and $b^n$ to
%compute $b^{n+1}$, then $b^{n+1}$ and $a^n$ to compute $a^{n+1}$.
%
%\glossentry[SMP]{Symmetric Multi-Processing (SMP)}
%Architecture with several processors having access to a shared main memory. They
%are controlled by a single OS and have full access to I/O.
%See
%\href{http://www.compunity.org/training/tutorials/2\%20Basic\_Concepts\_Parallelization.pdf}{this
%link} for a picture.
%

\newglossaryentry{die}
{
  name={Die},
  text={die},
  description={integrated circuits on a small piece of a semi-conducting
  material}
}
\newglossaryentry{stencil}
{
  name={Stencil},
  text={stencil},
  description={for numerical schemes, it indicates the number of adjacent cells
  needed to compute the given quantity}
}
\newglossaryentry{overhead}
{
  name={Overhead},
  text={overhead},
  description={happens when additional computer resources are needed to achieve a
  particular goal. For example, MPI overhead can happen for blocking operations,
  ,i.e a certain process loses time by waiting a request} 
}
\newglossaryentry{endianness}
{
  name={Endianess},
  text={endianness},
  description={Describes how the bytes of a word are ordered in the memory. In
  big-endian configurations, the most significant byte has the smallest address.
  .In little-endian, the least significant byte has the largest address}
}
