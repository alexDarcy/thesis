\chapter*[Conclusion]{Conclusion}
\addcontentsline{toc}{chapter}{Conclusion}

The goal of this thesis was to design and implement a framework for a global
atmospheric chemistry-transport model, adapted to massively parallel emerging
architectures. The characteristics of Pangolin have been chosen to obtain a
scalable application for current and future parallel architectures. In
particular, we focused on the message-passing paradigm applied to the advection
to drive our parallelization strategy.  We adopted an Eulerian approach for
advection on the sphere associated with a specific domain decomposition that
ease the parallelization of the code and appear competitive with the
semi-Lagrangian approaches.  In particular, the current CTM of M\'et\'eo-France,
MOCAGE, has a semi-Lagrangian advection scheme and the goal was to investigate
an alternative approach to exploit more easily the parallel architecture of the
large M\'et\'eo-France's computers. In an Eulerian framework, a common issue of
regular latitude-longitude grids is the convergence of meridians near the poles,
which reduces the time-step as the resolution increases and therefore strongly
penalizes the computational times. To bypass this `pole issue', we proposed a
new quasi-area-preserving grid where the number of cells decreases as we get
closer to the poles. This grid has quasi-uniform cell surfaces and preserves a
rectangular latitude-longitude structure.

As mass preservation is an important feature for the chemistry in a CTM, a
finite-volume approach was chosen for advection on that grid. Global and local
mass preservation is ensured by correcting the winds beforehand to ensure there
are divergence free. Choosing the order of the scheme proved to be a compromise
between accuracy and scability as higher-order schemes will increase the
communication volume. As a result, a second-order van Leer scheme was chosen to
favor scalability over accuracy. In practice, we expect to compensate the loss
of accuracy with finer grid resolutions. The current bi-dimensional advection
scheme and how it was adapted to our grid, along with general background
information, was discussed in Chapter~\ref{chap:transport}.

To validate the advection scheme adapted to the Pangolin grid, we examined key
properties for an advection scheme, such as stability, accuracy and mass
preservation (Chapter~\ref{chap:testing}). Pangolin was tested on algebraic test
cases and compared to other state-of-the-art transport schemes, using a recently
published testing suite. The other schemes were chosen such as to be also
second-order accurate and implemented on various grids. It was found that the
van Leer scheme implemented in Pangolin is in practice to be first-order.
This explained why Pangolin was most of the time less accurate than the models
selected here.  However, all the other schemes used a larger computational
stencil and the loss in accuracy was compensated in all of the tests with a
finer resolution. This is consistent with the initial design of the code where
moderate accuracy was chosen in order to improve parallel performances.

We have then examined the parallel performances of Pangolin using the MPI library
(Chapter~\ref{chap:parallel}). Due to our Eulerian approach, the parallelization
of the advection scheme is reduced to a domain decomposition problem. The global
grid is decomposed into connex subdomains of equal-area, which guarantees the
load balancing for advection. Instead of using a professional tool, we
developed our own partitioning algorithm to take advantage of the algebraic
features of the Pangolin grid. This choice was made to decrease the memory
footprint of Pangolin, in agreement with the possible future trends for computer
design as discussed in Chapter~\ref{chap:parallel}. The partitioning however
constrains the number of cores to achieve load-balancing. Even though the
constraint can be relaxed to accommodate other configurations, parallel
performances only improved for an optimal number of cores. We studied the
parallel performances of the 2D advection scheme up to 294 cores. The
scalability was found to be promising and it was estimated that addition of the
chemistry will largely improve it. We also examined the maximal number of cores
at a coarse resolutions in a strong-scaling study to find the limit of
parallelism of 2D advection. As expected, it was found that the parallel
performances decrease in extreme configurations where the subdomains are so
small that our strategy for hiding communication cost is no longer relevant.  With
such extremely small subdomains, only a very small portion of the memory of the
cores was used, making this configuration irrelevant for `real-life' simulations.

After validating the parallel advection, a linear ozone chemistry scheme was
added to examine the performances of Pangolin in `real-life' configurations and
compare it to MOCAGE simulations. Pangolin was run in an offline mode, where
wind and temperature data is forced every three hours. While MOCAGE has a 3D
advection scheme, advection in Pangolin is only bi-dimensional so it was
integrated using an isentropic vertical coordinate to make the comparison
relevant. It was found that generally, the large-scale structures were similar
and converge to the results of MOCAGE when the resolution is increased.

\subsubsection*{Perspectives}
Future versions of Pangolin will include an extension to 3D advection. The
addition of the vertical advection will make the wind corrections more
challenging. Also special care should be taken to obtain a non-divergent
advection consistent with the model providing the wind velocities.  Adding the
vertical advection will use the cores more effectively and as such improve the
parallel performances. At the moment, 2D advection is not `expensive' enough to
justify more than a few hundred of cores if the horizontal resolution at the
Equator is limited to around $1\degre\times1\degre$. If a complex chemistry scheme is
added, the chemical calculations should be expensive enough to greatly benefit
from the parallelization strategy adopted by Pangolin. Therefore, a larger
number of cores could be used in practice.

However, the addition of chemistry will most likely introduce load unbalances
due to several physical processes occurring not uniformly over the sphere
(heterogeneous chemistry, surface emissions, convection, vertical diffusion or
the day/night transitions).  Several strategies are possible to mitigate this
load unbalance, the most promising being the combination of OpenMP and MPI for a
hybrid parallel programming.

%-------------------------------------------------------------------------------
% French version
%-------------------------------------------------------------------------------

\selectlanguage{french}%
\chapter*{Conclusion}
\addcontentsline{toc}{chapter}{Conclusion (français)}

Le but de cette thèse était de concevoir et d'implanter une infrastructure
pour un modèle de chimie-transport atmosphérique global qui soit adapté aux
architectures massivement parallèles. Les caractéristiques de Pangolin ont été
choisies afin que le modèle passes à l'échelle sur les architectures parallèles
actuelles et futures. Nous nous sommes en particulier penchés sur le paradigme
de programmation à base d'échanges de messages en l'appliquant à l'advection,
ce qui a guidé notre stratégie de parallélisation. Un approche Eulérienne a été
retenue pour le schéma d'advection sur la sphère et combinée avec un nouvel
algorithme de décomposition de domaines. Cela a permis de faciliter la
parallélisation du code et semble pouvoir concurrencer les approches
semi-Lagrangiennes. En particulier, le modèle de chimie-transport de
Météo-France, MOCAGE, utilise un schéma semi-Lagrangien. L'objectif de la thèse
était d'examiner une approche alternative à MOCAGE pour exploiter les
supercalculateurs de Météo-France. Un problème classique de la grille régulière
latitude-longitude provient de la convergence des méridiens aux p\^oles, qui réduit
le pas de temps lorsque la résolution augmente et pénalise donc fortement le
temps de calcul. Pour éviter ce problème aux p\^oles, nous avons proposé une
nouvelle grille préservant approximativement les aires des cellules. Comme le nombre
de cellules décroît près des p\^oles, les surfaces des cellules sont quasiment
uniformes sur la sphère. La grille possède également une structure rectangulaire
latitude-longitude.

La conservation de la masse est une propriété important pour les MCTs, c'est
pourquoi une approche de type volume-finis a été utilisée sur notre grille. En
corrigeant les vents en amont pour rendre leur divergence nulle, on obtient
une conservation à la fois globale et locale de la masse. Le choix de l'ordre du
schéma d'advection implique de faire un compromis entre la précision et la
scalabilité car un ordre plus élevé augmentera le volume de communications.
C'est pourquoi un schéma du second-ordre de type van Leer a été retenu, afin de
privilégier la scalabilité. En pratique, la perte en précision sera compensée
par des résolutions plus fines. Le schéma 2D ainsi que la manière dont il a été
adapté à notre grille est présenté dans le chapitre~\ref{chap:transport}, avec
une présentation plus générale de la problématique du transport.

Afin de valider le schéma d'advection sur notre grille, nous avons examiné des
propriétés essentielles pour un schéma d'advection, à savoir la stabilité, la
précision et la préservation de la masse (chapitre~\ref{chap:testing}). De plus,
Pangolin a été testé avec des cas tests analytiques et comparé à d'autres
schémas de transport à l'aide d'un banc d'essai récemment publié. Les autres
schémas ont été choisis tels qu'ils soient également du second-ordre et qu'ils
soient implémentés sur différentes grilles. Les tests ont montré que le schéma
de type van Leer implémenté dans Pangolin était en pratique d'ordre un. Cela
explique pourquoi la plupart des tests ont montré que Pangolin était moins
précis. Ce résultat doit être nuancé par le fait que les autres modèles
utilisent un `halo' plus large. La différence en précision a été aussi
compensée avec une résolution plus fine, ce qui est cohérent avec les
hypothèses de conception où la scalabilité est privilégiée par rapport à la
précision.

Dans le chapitre~\ref{chap:parallel}, les performances en parallèle de Pangolin
à l'aide de la librairie MPI ont été étudiées. L'approche Eulérienne retenue
implique que la parallélisation de l'advection se ramène à un problème de
décomposition de domaines. Il consiste à découper la grille en sous-domaines
connexes de même taille afin de garantir l'équilibrage de la charge de calcul.
Au lieu d'utiliser des outils déjà existants, nous avons développé un nouvel
algorithme de partitionnement afin de prendre en compte les propriétés
analytiques de la grille. Ce choix provient de la volonté de diminuer le coût en
mémoire du modèle afin d'anticiper les tendances envisagées dans le
chapitre~\ref{chap:parallel}. Cependant, le partitionnement impose une
contrainte sur le nombre de c\oe{}urs pour que la charge de travail soit
équilibrée. La contrainte peut être assouplie mais les performances parallèles
ne seront améliorées que pour un nombre optimal de c\oe{}urs. Les performances en
parallèles ont été testées pour l'advection avec 294 c\oe{}urs. Les résultats sont
prometteurs et l'ajout de la chimie améliorera probablement fortement les
performances. Nous avons également examiné le nombre maximal de c\oe{}urs pour une
résolution grossière dans un test de \textit{strong-scaling} afin de déterminer la
limite du parallélisme pour l'advection 2D. Il a été confirmé que les
performances parallèles diminuent fortement quand les sous-domaines sont
extrêmement petits. De telles configurations rendent notre stratégie de
`recouvrement' des coûts de communications inefficace mais les sous-domaines
sont suffisament petits pour que la mémoire des c\oe{}urs soient très peu
utilisée. Ainsi, ces configurations ne correspondent pas à des situations
réalistes.

Afin d'effectuer une comparaison avec MOCAGE, un schéma linéaire pour l'ozone a
été ajouté pour examiner les performances de Pangolin dans des situations
réalistes. Pangolin a été utilisé en mode offline, où les vents et la
température sont forcés toutes les trois heures. Comme Pangolin ne dispose
actuelle que d'une advection 2D, nous avons utilisé des coordonnées isentropes
pour que la comparaison avec l'advection 3D de MOCAGE soit pertinente. Les
résultats ont montré que les structures les plus importantes sont bien
préservées et que Pangolin converge vers les résultats de MOCAGE lorsque la
résolution diminue.

\subsubsection*{Perspectives}
Une future version de Pangolin intègrera l'extension 3D pour l'advection.
L'ajout de l'advection verticale rendra la correction des vents plus délicate.
Il faudra aussi vérifier que l'advection non divergence soit cohérente avec le
modèle d'où proviennent les vents. Avec l'advection verticale, les c\oe{}urs seront
utilisés de manière plus efficace et cela améliorera les performances
parallèles. Pour l'instant, l'advection 2D ne justifie pas d'utiliser plus d'une
centaine de c\oe{}urs pour une résolution de $1\times1\degre$ à l'Équateur. Avec
une chimie complexe, la charge de calcul devrait être suffisamment importante
pour utiliser un nombre bien plus important de c\oe{}urs.

Cependant, l'ajout de la chimie s'accompagnera très probablement d'un
déséquilibre de la charge de calcul, due à la non-uniformité de certains
processus physiques (chimie hétérogène, émissions de surface, convection,
diffusion verticale ou la transition jour/nuit). Plusieurs stratégies sont
possibles pour y remédier mais la plus prometteuse semble être une combinaison
d'OpenMP et MPI pour une programmation parallèle hybride.
\selectlanguage{english}
