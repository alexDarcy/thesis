\newpage
\thispagestyle{empty} 

\begin{abstract}
We present in this thesis the development of a large-scale bi-di\-men\-sio\-nal
atmospheric transport scheme designed for parallel architectures with
scalability in mind. The current version, named Pangolin, contains a
bi-dimensional advection and a simple linear chemistry scheme for stratospheric
ozone and will serve as a basis for a future \gls{CTM}. For mass-preservation, a
van Leer finite-volume scheme was chosen for advection and extended to 2D with
operator splitting. To ensure mass preservation, winds are corrected in a
preprocessing step. We aim at addressing the "pole issue" of the traditional
regular latitude-longitude by presenting a new quasi-area-preserving grid
mapping the sphere uniformly. The parallelization of the model is based on the
advection operator and a custom domain-decomposition algorithm is presented here
to attain load-balancing in a message-passing context. To run efficiently on
current and future parallel architectures, algebraic features of the grid are
exploited in the advection scheme and parallelization algorithm to favor the
cheaper costs of flops versus data movement. The model is validated on algebraic
test cases and compared to other state-of-the-art schemes using a recent
benchmark. Pangolin is also compared to the CTM of M\'et\'eo-France, MOCAGE,
using a linear ozone scheme and isentropic coordinates.

\end{abstract}

\selectlanguage{french}%
\begin{abstract}
Cette thèse présente un modèle bi-dimensionnel pour le transport atmosphérique à
grande échelle, nommé Pangolin, conçu pour passer à l'échelle sur les achitectures
parallèles.  La version actuelle comporte une advection 2D ainsi qu'un schéma
linéaire de chimie et servira de base pour un modèle de chimie-transport (MCT).
Pour obtenir la conservation de la masse, un schéma en volume-finis de type van
Leer a été retenu pour l'advection et étendu au cas 2D en utilisant des
opérateurs alternés. La conservation de la masse est assurée en corrigeant les vents en
amont. Nous proposons une solution au problème "des pôles" de la grille
régulière latitude-longitude grâce à une nouvelle grille préservant
approximativement les aires des cellules et couvrant la sphère uniformément. La
parallélisation du modèle se base sur l'advection et utilise un algorithme de
décomposition de domaines spécialement adapté à la grille. Cela permet d'obtenir
l'équilibrage de la charge de calcul avec MPI, une librairie d'échanges de messages.
Pour que les performances soient à la hauteur sur les architectures parallèles
actuelles et futures, les propriétés analytiques de la grille sont exploitées
pour le schéma d'advection et la parallélisation en privilégiant le moindre coût
des flops par rapport aux mouvement de données. Le modèle est validé sur des cas
tests analytiques et comparé à des schémas de transport à l'aide d'un comparatif
récemment publié. Pangolin est aussi comparé au MCT de Météo-France via un schéma
linéaire d'ozone et l'utilisation de coordonnées isentropes.

\end{abstract}
\selectlanguage{english}

\newpage
